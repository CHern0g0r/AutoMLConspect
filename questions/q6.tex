\section{Мета-признаки}

\subsection*{Базовые метапризнаки}

\D{
    Мета-признаки =
    характеристики наборов данных.
    Присущи всем наборам данных для соответсвующей задачи.
}

Базовые мета-признаки для задачи классификации
\begin{itemize}
    \item число объектов
    \item число признаков
    \item число классов
    \item соотношение типов признаков
    \item число пропусков в данных
\end{itemize}

По базовым метода можно экспертно решать задачу выбора алгоритма:
\begin{itemize}
    \item можно оценивать время работы
    \item типы признаков влияют на выбора алгоритма
    \item число объектов может определять сложность модели
\end{itemize}

\D{
    Ивариант табличного набора данных

    При изменении порядка строк и столбцов в наборе данных
    скрытая зависимость, содержащаяся в нем не изменится.
}

Работа алгоритма должна не зависеть от порядка.

Большинство алгоритмов умеют поддерживать инвариант.

Некоторые его не сохраняют, но могут случайно перемешивать
входные данные.

\subsection*{Статистические метапризнаки}

Агрегация:
\begin{itemize}
    \item Каждый столбец агрегируется как множество чисел
    \item Результаты для всех столбцов агрегируются в одно число
    \item Комбинируя различные функции можно получать различные
    метапризнаки
\end{itemize}

Функции агрегации:
\begin{itemize}
    \item min
    \item max
    \item matoj
    \item std
    \item асимметрия
    \item эксцесс
\end{itemize}

Мета-признаки должны быть инвариантны к изменению единиц измерений,
поэтому в большинстве случаев значения фичи должны быть нормализованы.

\subsection*{Метапризнаки категорий}

\begin{itemize}
    \item Число категорий.
    \item Вероятность встретить конкретное значение в категории.
    \item Энтропия вероятностей.
\end{itemize}

\subsection*{Задача обучения с учителем}

В задаче обучения с учителем требуется сохранять информацию
о целевом признаке.

Можно агрегировать целевой признак.

Можно оценивать взаимосвязь двух столбцов. (

    Этнропия = $\sum_i p(i) \sum_j p(c_j | f(x) = i) \log p(c_j | f(x) = i)$,
    Джини.
    Всякие критерии, условные статистики, корреляции

)

Обычно оцениваются пары всех признкаов с целевым, но можно
агрегировать признаки между собой и получать квадрат различных
значений.

\subsection*{Структурные метапринаки}

Обучаем модель на наборе данных и использовать параметры
ее структуры в качестве метапризнаков.

Модель зависит от датасета, поэтому ее структура описывает
датасет.

Инвариант сохраняется т.к. модель сохраняет.

Структурные признаки для дерева:
\begin{itemize}
    \item Глубина листа, вершины
    \item Число объектов в листе
    \item ...
\end{itemize}

Еще:
\begin{itemize}
    \item Коэффициенты линейной модели
    \item Коэффициенты нейронки
    \item ...
\end{itemize}

\subsection*{Лэндмарки}

\D{
    В качестве метапризнака берем значение функции качества
    какой-то модели на датасете.
}